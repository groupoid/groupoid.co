\usepackage[osf]{mathpazo}
\usepackage{musicography}
\usepackage{epigraph}
\usepackage[mathcal,mathbf]{euler}
\usepackage{amsmath,amssymb,amsthm}
\usepackage{graphicx,sidecap,tikz}
\usepackage{siunitx}
\usepackage{fullwidth}
\usepackage{fontspec}
\usepackage{hyphenat}
\usepackage{url}
\usepackage{ifthen}
\usepackage{tikz-cd}
\usepackage[english,russian]{babel}
\usepackage{bussproofs}
\usepackage{tabstackengine}
\usepackage{graphicx}
\usepackage{cite}
\usepackage{moreverb}
\usepackage{listings}
\usepackage{amssymb}
\usepackage{mathtools}
\usepackage[only,llbracket,rrbracket,llparenthesis,rrparenthesis]{stmaryrd}

\usetikzlibrary{matrix}
\usetikzlibrary{babel}
\theoremstyle{definition}
\newtheorem{theorem}{Теорема}
\newtheorem{definition}{Визначення}
\newtheorem{exercise}{Вправа}
\newtheorem{example}{Приклад}

\newcommand*{\incmap}{\hookrightarrow}

\def\mapright#1{\xrightarrow{{#1}}}
\def\mapleft#1{\xleftarrow{{#1}}}
\def\mapup#1{\Big\uparrow\rlap{\raise2pt{\scriptstyle{#1}}}}
\def\mapupl#1{\Big\uparrow\llap{\raise2pt{\scriptstyle{#1}}}}
\def\mapdown#1{\Big\downarrow\rlap{\raise2pt{\scriptstyle{#1}}}}
\def\mapdownl#1{\Big\downarrow\llap{\raise2pt{\scriptstyle{#1}}}}
\def\mapdiagl#1{\vcenter{\searrow}\rlap{\raise2pt{\scriptstyle{#1}}}}
\def\mapdiagr#1{\vcenter{\swarrow}\rlap{\raise2pt{\scriptstyle{#1}}}}

\lefthyphenmin=1
\hyphenpenalty=100
\tolerance=3000

\newfontfamily{\cyrillicfont}{Times New Roman}
\setmainfont{Times New Roman}

\addto\captionsrussian{\renewcommand{\contentsname}{Зміст}}
\addto\captionsrussian{\renewcommand{\bibname}{Список використаних джерел}}
\addto\captionsrussian{\renewcommand{\chaptername}{Розділ}}
\addto\captionsrussian{\renewcommand{\tablename}{Таблиця}}

\lstset{morekeywords={record,data,inductive,extend,enum,
Type,Path,unit,Unit,Nat,List,let,in,eq,type,sh,sub,star,
var,dep,norm,fun,app,lambda,arrow,pi,case,receive,spawn,
send,name,list,nat,tele,case,id,sigma,pair,fst,snd,id,idPair,idJ,
branch,label,ctor,sum,prod,ty,lam,split,Glue,glue,unglue,Pi,
Sigma,pr1,pr2,Beta,Eta,Beta1,Beta2,Eta2,refl,ap,apply,apd,J,
process,execute,storage,axiom,undefined,comp,PathP,fill,
precategory,catfunctor,equiv,trans,base,loop,S1,S2,pub,
struct,Vec,Arc,Cell,sub,rcv,snd,spawn,module,import,
coerce,cong,Fixpoint,Parameter,Definition,CoInductive,
Require,Import,CoFixpoint,remote,ac,total,fix,ext,fn,
call,true,false,is_defined,datum,id_intro,id_elim,
where,bool,maybe,either,unit,empty,list,stream,vector,fin}}

\lstset{
    backgroundcolor=\color{white},
    keywordstyle=\color{blue},
    basicstyle=\bf\ttfamily\footnotesize,
    xleftmargin=0cm,
    columns=fixed}

\newcommand{\titleVAK}{
\begin{titlepage}
\begin{center}
\thispagestyle{empty}
\large Міністерство освіти і науки України \\
Національний технічний університет України
"Київський політехнічний інститут імені Ігоря Сікорського"
\vspace{0.5cm}
\begin{flushright}
\small Кваліфікаційна наукова праця \\
на правах рукопису
\end{flushright}
\vspace{0.5cm}
\Large Сохацький Максим Еротейович \\
\vspace{0.5cm}
\begin{flushright}
\small УДК 510.2---510.6
\vspace{1cm}
\end{flushright}
\Large Дисертація \\
\large Концептуальна модель системи доведення теорем \\ на основі гомотопічної теорії типів \\
\vspace{0.5cm}
\begin{flushright}
\small Спеціальність 01.05.03 --- математичне та програмне забезпечення \\ обчислювальних машин і систем
\end{flushright}
\vspace{1cm}
\begin{flushleft}
\renewcommand{\baselinestretch}{1.0}
\small Подається на здобуття наукового ступеня доктора філософії. Дисертація містить результати власних
дослідженя, використання ідей, результатів і текстів інших авторів мають посилання на джерела.
\renewcommand{\baselinestretch}{1.5}
\end{flushleft}
\vspace{0.5cm}
\begin{flushright}
\small Науковий консультант: Павло Павлович Маслянко
\end{flushright}
\vspace{0.5cm}
Київ --- 2018
\end{center}
\end{titlepage}
}

\newcommand{\embedbib}{
\cite{Lof72}
\cite{Lof84}
\cite{Coq88}
\cite{Hofmann96}
\cite{Henk93}
\cite{Erik97}
\cite{Hermida95}
\cite{Curien08}
\cite{MacLane71}
\cite{Lawvere09}
\cite{Dybjer08}
\cite{Clairambault05}
\cite{Abel08}
\cite{Seely84}
\cite{Curien14}
\cite{Castellan14}
\cite{Voevodsky14}
\cite{Dybjer95}
\cite{Bishop67}
\cite{Nordstrom90}
\cite{Hermida98}
\cite{Barthe00}
\cite{Voevodsky15}
\cite{Sozeau}
\cite{Selsam16}
\cite{Bohm85}
\cite{Pfenning89}
\cite{Wadler90}
\cite{Gambino03}
\cite{Dybjer94}
\cite{Jacobs97}
\cite{Vene00}
\cite{Basold16}
\cite{Hofmann94}
\cite{Jacobs99}
\cite{Joyal14}
\cite{HoTT13}
\cite{Mortberg17}
\cite{Shulman15}
\cite{Orton17}
\cite{Huber16}
\cite{Huber17}
\cite{Angiuli16}
\cite{Angiuli162}
}

\newcommand{\titleINF}{
\begin{titlepage}
\thispagestyle{empty}
\begin{flushleft}
УДК 510.2
\end{flushleft}
\begin{flushright}
{Максим Сохацький}\\
\vspace{0.5cm}
{\LARGE \bf Мова простору \\ }
\vspace{2cm}
15 жовтня 2018 e8020d8141ece25695efc185aec732f35867386b \\
11 живтня 2018 64c2e49765d3b997d68999c88448fef7136079d4
\vspace{0.3cm}
\url{http://bit.ly/groupoid}
\end{flushright}
\end{titlepage}
}
