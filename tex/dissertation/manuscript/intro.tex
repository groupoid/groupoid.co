\chapter*{ВСТУП}
\addcontentsline{toc}{chapter}{ВСТУП}
\addcontentsline{toc}{section}{Актуальність роботи}
\addcontentsline{toc}{section}{Формалізована постановка задачі}
\addcontentsline{toc}{subsection}{Об'єкт та предмет дослідження}
\addcontentsline{toc}{subsection}{Мета і задачі дослідження}
\addcontentsline{toc}{subsection}{Методи дослідження}
\addcontentsline{toc}{section}{Практичні результати}
\addcontentsline{toc}{section}{Структура роботи}
\addcontentsline{toc}{subsection}{Формальна верифікація}
\addcontentsline{toc}{subsection}{Середовище виконання та мови програмування}
\addcontentsline{toc}{subsection}{Базова бібліотека}
\addcontentsline{toc}{subsection}{Математика}
\addcontentsline{toc}{subsection}{Додатки}

\epigraph{Присвячується Маші та Міші}{}

\paragraph{}
Тут дамо тему, предмет та мету роботи, а також дамо опис структури роботи.

\section*{Актуальність роботи}
Ціна помилок в індустрії надзвичайно велика. Наведемо
відомі приклади: 1) Mars Climate Orbiter (1998), помилка невідповідності
типів британської метричної системи, коштувала 80 мільйонів фунтів стерлінгів.
Невдача стала причиною переходу NASA повністю на метричну систему в 2007 році.
2) Ariane Rocket (1996), причинан катастрофи -- округлення 64-бітного дійсного
числа до 16-бітного. Втрачені кошти на побудову ракети та запуск 500 мільйонів
3) Помилка в FPU в перших Pentium (1994), збитки на 300 мільйонів.
4) Помилка в SSL (heartbleed), оцінені збитки у розмірі 400 мільйонів.
5) Помилка у логіці бізнес-контрактів EVM та
DAO (неконтрольована рекурсія), збитки 50 мільйонів.
Більше того, і найголовніше, помилки у програмному забезпеченні можуть
коштувати життя людей.

\section*{Формалізована постановка задачі}
Задачою цього дослідження є побудова мінімальної системи
мовних засобів для побудови ефективного циклу верифікації програмного
забезпечення та доведення теорем. Основні компоненти системи, як продукт
дослідження: 1) інтерпретатор безтипового лямбда числення;
2) компактне ядро --- система з однією аксіомою;
3) мова з індуктивними типами;
4) мова з гомотопічним інтервалом $[0,1]$;
5) уніфікована базова бібліотека.

\subsection*{Об'єкт та предмет дослідження}
Об'єктом дослідження данної роботи є: 1) системи верифікації
програмного забезпечення; 2) системи доведення теорем 3) мови програмування
4) операційні системи, які виконують
обчислення в реальному часі; 3) їх поєднання, побудова формальної системи для
унифікованого середовища, яке поєднує середовище виконання та систему
верифікації у єдину систему мов та засобів.

Предметом дослідження такої системи мов є теорія типів, яка вивчає обчислювальні властивості мов.
Теорія типів виділилася в окрему науку Пером Мартіном-Льофом як запит на вакантне місце у
трикутнику теорій, які відповідають ізоморфізму Каррі-Говарда-Ламбека (Логіки, Мови, Категорії).
Інші дві це: теорія категорій та логіка вищих порядків.

\subsection*{Мета і задачі дослідження}
Одна з причина низького рівня впровадження у виробництво систем
верифікації -- це висока складність таких систем. Складні системи
верифікуються складно. Ми хочемо запропонувати спрощений
підхід до верифікації -- оснований на концепції компактних
та простих мовних ядер для створення специфікацій, моделей,
перевірки моделей, доведення теорем у теорії типів з кванторами.

Формалізація семантики відбувається завдяки теорії категорій,
яка є абстрактною алгеброю фунцій, метематичним інструментом
для формалізації мов програмування та довільних
математичних теорій які описуються логіками вищих порядків.

Завдання цього дослідження є побудова єдиної системи, яка поєднує середовище
викодання та систему верифікації програмного забезпечення. Це прикладне дослідження,
яке є сплавом фундаментальної математики та інженерних систем з формальними методами верицікації.

\subsection*{Методи дослідження}

Незалежно від піходу до верифіції, формальна верифікація неможлива,
якщо мова програмування моделі формально не визначена. Це означає шо значна міра
програмного забезпечення може бути автоматично верифікована тільки для тих мов,
формальні моделі яких побудовані, на даний момент це тільки мова С.
Більше того, не завжди можна також формально довести те, що програма завершиться,
потрібно звужувати клас програм, якщо формальні специфікації містять такі властивості.

\section*{Практичні результати}

\section*{Структура роботи}

Якшо коротко суть роботи зводиться до побудови системи, яка складається з:
i) середовища виконання; ii) формального інтерпретатора; iii) системи формальних мов
для доведення теорем математики, програмної інженерії та філософії.

\subsection*{Формальна верифікація}

У розділі 1 дається огляд існуючих рішень для доведення
властивостей систем та моделей, класифікуються мови програмування
та системи доведення теорем.

\subsection*{Середовище виконання та мови програмування}

У розділі 2 розглядається повний стек формального програмного забезпечення
від віртуальної машини, байт-код інтерпретатора та середовища виконання
та планування процесів до формальної мови для доведення теорем (або сімейства мов).

\subsection*{Базова бібліотека}

У розділі 3 описується базова бібліотека, написана на самій потужній
формальній мові системи доведення теорем.

\subsection*{Математика}

У розділі 4 пропонується ряд математичних моделей та теорій з використанням
базової бібліотеки розділу 3 та мови гомотопічної системи типів.

\subsection*{Додатки}

У додатках надаються приклади іншого використання фомальних мов та моделей,
зокрема для мінімальної формальної мови, побудованої в рамках дисертації,
та мови програмування Coq. А також дається приклад використання
гомотопічної мови для формальної філософії.

