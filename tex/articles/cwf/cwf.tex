% copyright (c) 2018 Groupoid Infinity

\documentclass{article}
\usepackage{amscd}
\usepackage{listings}
\usepackage{graphicx}
\usepackage{amsmath}
\usepackage{mathtools}
\usepackage{amssymb}
\usepackage{amsthm}
\usepackage{url}
\usepackage{tikz-cd}
\usepackage[utf8]{inputenc}

\theoremstyle{definition}
\newtheorem{definition}{Definition}
\newtheorem{theorem}{Theorem}

\newcommand*{\thead}[1]{\multicolumn{1}{c}{\bfseries #1}}
\lstset{basicstyle=\small,inputencoding=utf8}

\begin{document}

\title{Types VIII: Contextual Categories \\ for Martin-Löf Type Theory}
\author{Maksym Sokhatskyi $^1$}
\date{ $^1$ National Technical University of Ukraine \\
       \small Igor Sikorsky Kyiv Polytechnical Institute \\
       \today }

\maketitle

\begin{abstract}

Categorical Semantics of Dependent Type Theory.
\\
\\
{\bf Keywords}: Formal Methods, Type Theory, Programming Languages,
          Theoretical Computer Science, Applied Mathematics,
          Cubical Type Theory, Martin-Löf Type Theory
\end{abstract}
\tableofcontents

\newpage
\section{Categories with Families}

Here is a short informal description of
categorical semantics of dependent type theory given by Peter Dybjer.
The code is by Thierry Coquand ported to cubical by 5HT
\footnote{http://www.cse.chalmers.se/~peterd/papers/Ise2008.pdf}.

\begin{definition} (Fam). Fam is the category of families
of sets where objects are dependent function
spaces $(x:A)\rightarrow B(x)$ and morphisms with domain
$Pi(A,B)$ and codomain $Pi(A',B')$ are pairs of
functions $<f:A\rightarrow A',g(x:A):B(x)\rightarrow B'(f(x))>$.
\end{definition}

\begin{definition} ($\prod$-Derivability).
$\Gamma\vdash A = (\gamma:\Gamma)\rightarrow A(\gamma)$.
\end{definition}

\begin{definition} ($\sum$-Comprehension).
$\Gamma;A = (\gamma:\Gamma)*A(\gamma)$. Comprehension is not assoc.
$$
    \Gamma;A;B \neq \Gamma;B;A
$$
\end{definition}

\begin{definition} (Context).
The C is context category where objects are contexts and morphisms are substitutions.
Terminal object $\Gamma=0$ in C is called empty context.
Context comprehension operation $\Gamma;A = (x:\Gamma)*A(x)$ and
its eliminators: $p:\Gamma;A\vdash\Gamma$, $q:\Gamma;A\vdash A(p)$ such that
universal property holds: for any $\Delta:ob(C)$, morphism $\gamma:\Delta\rightarrow\Gamma$,
and term $a:\Delta\rightarrow A$ there is a unique morphism $\theta=<\gamma,a>:\Delta\rightarrow\Gamma;A$
such that $p\circ\theta=\gamma$ and $q(\theta)=a$. Statement. Subst is assoc.
$$
    \gamma(\gamma(\Gamma,x,a),y,b) = \gamma(\gamma(\Gamma,y,b),x,a)
$$
\end{definition}

\begin{definition} (CwF-object).
A CwF-object is a $Sigma(C,C\rightarrow Fam)$
of context category C with contexts as objects and
substitutions as morphisms and functor $T:C\rightarrow Fam$ where
object part is a map from a context $\Gamma$ of C to famility
of sets of terms $\Gamma\vdash A$ and morphism part is a map from
substitution $\gamma:\Delta\rightarrow\Gamma$ to a pair of functions which perform
substitutions of $\gamma$ in terms and types respectively.
\end{definition}

\begin{definition} (CwF-morphism).
Let $(C,T):ob(C)$ where $T:C\rightarrow Fam$.
A CwF-morphism $m: (C,T)\rightarrow(C',T')$ is a pair $<F:C\rightarrow C',\sigma:T\rightarrow T'(F)>$
where F is a functor and $\sigma$ is a natural transformation.
\end{definition}

\begin{definition} (Category of Types).
Let we have CwF with (C,T) objects
and $(C,T)\rightarrow(C',T')$ mophisms. For a given context $\Gamma$ in Ob(C) we can
construct a $Type(\Gamma)$ -- the category of types in context $\Gamma$ with
set of types in contexts as objects as and functions
$f:\Gamma;A\rightarrow B(p)$ as morphisms.
\end{definition}

\begin{definition} (Local Cartesian Closed Category).
$$
    LCCC(C) = \prod_{x:C}IsCCC(C_{/x}).
$$
Usually, from the mathematical point of view, there are no differences between
different syntactic proofs of the same theorem. However, from the programming point of view, we can think of code reuse and precisely defined math libraries that reduce
the overall code size and provide simplicity for operating complex
structures (most heavy one is the categorical library). So in this work, we will focus
on proper decoupling and more programming friendly base library still usable
for mathematicians.
\end{definition}

\bibliographystyle{plain}
\bibliography{cwf}

\end{document}

