% copyright (c) 2018 Groupoid Infinity

\documentclass{article}
\usepackage{listings}
\usepackage{amsmath}
\usepackage{amssymb}
\usepackage{amsthm}
\usepackage{url}
\usepackage{tikz-cd}
\usepackage[utf8]{inputenc}

\theoremstyle{definition}
\newtheorem{definition}{Definition}
\newtheorem{theorem}{Theorem}
\newcommand*{\incmap}{\hookrightarrow}
\newcommand*{\thead}[1]{\multicolumn{1}{c}{\bfseries #1}}
\lstset{basicstyle=\small,inputencoding=utf8}

\begin{document}

\title{Issue VI: Set Theory}
\author{Maxim Sokhatsky $^1$}
\date{ $^1$ National Technical University of Ukraine \\
       \small Igor Sikorsky Kyiv Polytechnical Institute \\
       \today }

\maketitle

\begin{abstract}
CW-complexes are fundamental objects in homotopy type theory
and even included inside cubical type checker in a form of
higher (co)-inductive types (HITs).
Just like regular (co)-inductive types could be described as recursive
terminating (well-founded) or non-terminating trees,
higher inductive types could be described as CW-complexes.
Defining HIT means to define some CW-complex
directly using cubical homogeneous composition structure as an
element of initial algebra inside cubical model.
\\
\\
{\bf Keywords}: Set Theory, Cubical Type Theory
\end{abstract}
\tableofcontents

\newpage
\section{Prerequisites}

\subsection{ETCS}
Lawvere’s theory ETCS (Lawvere 2005) has eight axioms:
(L1) finite roots exist,
(L2) the exponential of any pair of objects exist,
(L3) there is a Dedekind-Peano object,
(L4) the terminal object is separating,
(L5) axiom of choice,
(L6) every object not isomorphic to an initial object contains an element,
(L7) Each element of a sum is a member of one of its injections,
(L8) there is an object with more than one element.


\bibliographystyle{plain}
\bibliography{cwf}

\end{document}

