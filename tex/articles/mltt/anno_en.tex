\documentclass{article}
\usepackage{microtype}
\usepackage{titlesec}
\usepackage[absolute]{textpos}
\usepackage{amsmath}
\usepackage{cite}
\usepackage[strict=true,style=english]{csquotes}
\setquotestyle{english}
\usepackage{amssymb}
\usepackage{amsthm}
\usepackage{url}
\usepackage{tikz}
\usepackage{tikz-cd}
\usetikzlibrary{matrix}
\usepackage[utf8]{inputenc}
\usepackage[english,russian]{babel}
\usetikzlibrary{babel}
\usepackage{listings}
\usepackage[tableposition=top]{caption}
\theoremstyle{definition}
\newtheorem{theorem}{Theorem}
\newtheorem{definition}{Definition}
\newtheorem{exercise}{Exercise}
\newtheorem{example}{Example}
\captionsetup[table]{labelformat=empty}
\lstMakeShortInline[mathescape=true,columns=fixed]|
\def\mapright#1{\xrightarrow{{#1}}}
\def\mapleft#1{\xleftarrow{{#1}}}
\def\mapup#1{\Big\uparrow\rlap{\raise2pt{\scriptstyle{#1}}}}
\def\mapupl#1{\Big\uparrow\llap{\raise2pt{\scriptstyle{#1}}}}
\def\mapdown#1{\Big\downarrow\rlap{\raise2pt{\scriptstyle{#1}}}}
\def\mapdownl#1{\Big\downarrow\llap{\raise2pt{\scriptstyle{#1}}}}
\def\mapdiagl#1{\vcenter{\searrow}\rlap{\raise2pt{\scriptstyle{#1}}}}
\def\mapdiagr#1{\vcenter{\swarrow}\rlap{\raise2pt{\scriptstyle{#1}}}}
\lstset{basicstyle=\small,inputencoding=utf8}
\setlength{\TPHorizModule}{4mm}
\setlength{\TPVertModule}{\TPHorizModule}
\textblockorigin{4mm}{4mm}
\setlength{\parindent}{0pt}
\titleformat{\section}[block]{\Large\bfseries\filcenter}{}{1em}{}

\begin{textblock}{3}(2,3)
\small \text{УДК 510.2:510.6}
\end{textblock}

\begin{document}

\title{\small Максим M.E. Sokhatsky^{*}}
\author{Issue I: Internalizing Martin-Löf Type Theory}
\date{ \small Igor Sikorsky Kyiv Polytechnical Institute, Kyiv, Ukraine\\
       \small $^*$Corresponding author: maxim@synrc.com}

\maketitle

\section*{Annotation}
\hyphenpenalty=0

\begin{abstract}

\justifying

This article demonstrates formal Martin-Löf Type Theory (MLTT)
embedding into the host type system with constructive proofs
of the complete set of inference rules. This was recently made
possible by cubical type theory and cubical type
checker cubicaltt{\bf cubicaltt}\footnote{http://github.com/mortberg/cubicaltt} in 2017.
The long road from pure type systems of AUTOMATH by de Bruijn to type
checkers with homotopical core was made. This article touches only
the formal MLTT core type system with $\Pi$ and $\Sigma$ types (that correspond
to $\forall$ and $\exists$ quantifiers for mathematical reasoning) and identity type.

Each language implementation needs to be checked. One of the possible
test cases for type checkers is the direct embedding of type theory
model into the language of the type checker. As types are formulated
using 5 types of rules (formation, intro, elimination,
computation, uniqueness), we construct aliases for the host
language primitives and use type checker to prove that it has
the realization of MLTT. This could be seen as an ultimate test
sample for type checker as intro-elimination fusion resides in
beta-eta rules, so by proving them, we prove properties of the
host type checker.

More formally cubical MLTT internalization proofs the J eliminator
and its beta rule, something that was impossible without geometrical
realization of cubical type theory. Also, this issue opens a series
of articles dedicated to the formalization of the foundations
of mathematics in cubical type theory, MLTT modeling and cubical
proof verification. As many may not be familiar with  types,
this issue presents different interpretations of core types from
other areas of mathematics.

We should note that this is an entrance to the internalization
technique, and after formal MLTT embedding, we could go through
inductive types up to embedding of CW-complexes as the indexed
gluing of the higher inductive types.
\\
\\
\\
\end{abstract}
\end{document}